\chapter{Introduction}
\label{sec:introduction}
This dissertation develops a data-oriented software architecture for the simulation of space systems with the goal of improving the conceptual and computational complexity of spacecraft simulators.

Simulations are an essential tool throughout the lifecycle of a space system.
These range from simple astrodynamics simulations in the design phase to full operational simulators running emulated \ac{obsw} in the operational phases.
The latter category are soft real-time systems that can by 

\todo{short solution description}

This introduction describes the motivation for this dissertation and the statement of the problem in section \ref{sec:motivation}.
Based on the problem statement the scientific objective of this dissertation is defined in section \ref{sec:scientific-objective}.
The chapter concludes in section \ref{sec:outline} with an outline of the remainder of this dissertation.

\section{Motivation}
\label{sec:motivation}

\section{Scientific Objective}
\label{sec:scientific-objective}

\section{Outline}
\label{sec:outline}
The scientific examination of the subject of this dissertation as described above begins with a discussion of the state of the art in \todo{topic} in chapter \ref{sec:state-of-the-art}.

In chapter \ref{sec:requirements} the requirements for the \topic are developed based on the state of the art.


