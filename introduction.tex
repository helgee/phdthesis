\chapter{Introduction}
\label{sec:introduction}
Within this dissertation the \ac{kml} is introduced.
The motion of an object in space is governed by non-linear equations.
Numerical analysis is therefore indispensable for the design of a spacecraft's trajectory.
Since the early days of spaceflight in the middle of the 20\textsuperscript{th} century software tools have been used to conduct the required numerical integrations and optimisations \todo{citation needed}.
The mission design and the design of the space system, which includes the space-based components as well as the ground-based components, are highly dependent on each other.
During the 
Commercial missions
Scientific space missions on the other hand are often extending the boundaries of what is technically feasible and are generally more complex than commercial missions.
Thus the mission design needs to take the unique properties and constraints of these missions into account.

\todo{short problem description}

\todo{short solution description}

This introduction describes the motivation for this dissertation and the statement of the problem in section \ref{sec:motivation}.
Based on the problem statement the scientific objective of this dissertation is defined in section \ref{sec:scientific-objective}.
The chapter concludes in section \ref{sec:outline} with an outline of the remainder of this dissertation.

\section{Motivation}
\label{sec:motivation}

\section{Scientific Objective}
\label{sec:scientific-objective}

\section{Outline}
\label{sec:outline}
The scientific examination of the subject of this dissertation as described above begins with a discussion of the state of the art in \todo{topic} in chapter \ref{sec:state-of-the-art}.

In chapter \ref{sec:requirements} the requirements for the \topic are developed based on the state of the art.


