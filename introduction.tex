\chapter{Introduction}
\label{sec:introduction}
This dissertation develops a data-oriented software architecture for the simulation of space systems with the goal of improving the conceptual and computational complexity of spacecraft simulators.

Simulations are an essential tool throughout the lifecycle of a space system.
These range from simple astrodynamics simulations in the design phase to full operational simulators running emulated \ac{obsw} in the operational phases.
Operational simulators are used to train spacecraft operators and validate control procedures by providing a digital twin of satellite.
To provide realistic feedback they need to achieve high modelling fidelity while at the same time fulfilling soft real-time requirements.
A simulator for a single spacecraft can easily use the available hardware resources on a high-performance commodity server to capacity.
Simulating constellations of satellites such as the Galileo navigation constellation which will consist of 30 satellites when it is fully deployed requires distributing the simulation across several machines.\citeme
Future constellation such as the Starlink constellation\citeme are planned to consist of tens of thousands of satellites and are thus called \enquote{megaconstellations}.
Based on the performance characteristics of today's operational simulators, it is very likely that it will be prohibitively expensive to simulate a full megaconstellation with the current simulator technology.

Another technology domain with similar performance requirements that also models physical environments are \ac{ris} such as \ac{vr} and \ac{ar} systems.

\todo{short solution description}

This introduction describes the motivation for this dissertation and the statement of the problem in section \ref{sec:motivation}.
Based on the problem statement the scientific objective of this dissertation is defined in section \ref{sec:scientific-objective}.
The chapter concludes in section \ref{sec:outline} with an outline of the remainder of this dissertation.

\section{Motivation}
\label{sec:motivation}

\section{Scientific Objective}
\label{sec:scientific-objective}

\section{Outline}
\label{sec:outline}
The scientific examination of the subject of this dissertation as described above begins with a discussion of the state of the art in \todo{topic} in chapter \ref{sec:state-of-the-art}.

In chapter \ref{sec:requirements} the requirements for the \topic are developed based on the state of the art.


