\chapter{Introduction}
The continuous miniaturization of computing devices has led to an age of ubiquituous computing.
Fast wireless data connections through mobile networks have become the norm in industrialized countries.
Mobile phones have turned into always-connected, powerful special purpose computers.
More and more companies embed small computing systems with broadband connections into their products, turning everyday items into connected devices.
Equipped with sensors and actuators they can build an accurate digital model of their surroundings and interact with their environment.
By bridging the digital and the physical world \ac{cps} are created which are the foundation of the \ac{iot}.

The rise in computing power, that was predicted correctly for decades by Moore's law, has enabled \ac{it} companies like Google and Facebook to deploy tremendous computing power based on cheap commodity hardware.
Outsourcing computation and data storage to the data centres of these companies, moving applications to the \enquote{cloud}, has become a cost-efficient way for many businesses to provide web services to their customers.
Cloud computing also provides the necessary infrastructure to analyze and derive value from the large amounts of data that are being created in these web services.
Large scale data analysis and machine learning systems are being deployed to make sense of this \enquote{big data}.
The ever-growing quantity of data has outrun the gains in processing power through Moore's law.
Data analysis systems therefore cannot scale only vertically by increasing processing power but must also scale horizontally by forming computing clusters made up of multitudes of distributed machines.
This has profound implications for the data analysis software which must conform to the hardware architecture of the cluster.
The microservice software architecture addresses this issue by splitting monolithic sytems into minimal web services.
By utilizing container-based virtualization technology the microservices and their dependencies are packaged into self-contained images.
Multiple instances of each microservice can then be deployed on any node of the cluster.

Turning the machine tools of the manufacturing industry into \ac{cpps} and integrating them into intelligent production networks is expected to produce significant amounts of data as well.
Analyzing the data generated by the sensors and actuators that are part of the \ac{cpps} many new possibilites open up.
These include predictive maintenance, efficiency improvements, better quality control and self-optimizing systems.
Since \ac{cpps} need to react in real-time to outside events and the computing power of their embedded processors is constrained, processing the generated data at the source is generally not feasible.
The software systems within a CPPS-equipped factory must also be able to adapt to rapid changes in the manufacturing environment.

Within this dissertation a service-oriented architecture for the ad-hoc creation and deployment of data analysis pipelines on heterogenuous computing hardware is developed.
It consists of container-based microservices connected through asynchronous and synchronous communication channels deployed to hybrid clusters containing nodes within private and public cloud computing environments.
They collect data from \ac{cpps} sensors, analyze it, and use the results to monitor and modify the parameters of the production systems.

