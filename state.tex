\chapter{State of the Art}
\label{sec:state-of-the-art}

In this chapter the state of the art in spacecraft trajectory optimisation is discussed.
Section~\ref{sec:spacecraft-trajectory-optimisation} gives a general introduction to \sto before introducing the mathematical formulation of the \sto problem and the common solution methods.
In section~\ref{sec:software-systems-for-spacecraft-trajectory-optimisation} 

\section{Spacecraft Trajectory Optimisation}
\label{sec:spacecraft-trajectory-optimisation}
Spacecraft trajectory optimisation can be described as
\begin{quote}
    \ldots the determination of a trajectory for a spacecraft that satisfies specified initial and terminal conditions, that is conducting a required mission, while minimizing some quantity of importance \autocite[p. 1]{conway-problem-2014}.
\end{quote}

These initial conditions are usually determined by the combination of initial orbit and spacecraft mass that the selected launch vehicle can achieve.
Whereas the terminal conditions are the operational orbits around the Earth or other celestial bodies that the spacecraft is supposed to reach during its lifetime.
While on an operational orbit and for most missions also during transit its trajectory must enable the spacecraft to fulfill its mission objectives and operational constraints.
Mission objectives might be conducting observations and measurements with scientific instruments, providing communication services, or transporting humans safely through space.
Regular opportunies for contact with the ground stations and mission control systems on Earth are common operational constraints.

The abovementioned quantities of importance are in most cases propellant consumption and transfer time.
Since propellant accounts for a large percentage of the highly constrained initial launch mass every gram of propellant spared can be utilised for the payload of the spacecraft.
The payload on the other hand is the reason that a space mission is conducted in the first place.
More available mass means more room for instruments, transponders etc.\ and therefore improved mission performance.
Hence, the objective is to \enquote{maximize the fraction of the spacecraft that is not devoted to propellant} \autocite[p. 1]{conway-problem-2014}.

Because it is generally possible to find a trajectory with decreased propellant consumption by increasing the transfer time an upper time limit needs to be imposed \autocite[p. 1]{conway-problem-2014}.
This is especially important for interplanetary missions where transfer times can approach decades.
The \ac{esa} spacecraft Rosetta for example was launched on March 4, 2004 and only reached its target orbit 10 years later on August 6, 2014 \autocite{glassmeier-rosetta-2007}.
Thus the optimal trajectory is usually a trade-off between propellant requirements and transfer time.

\subsection{The Trajectory Optimisation Problem}
\label{sec:trajectory-optimisation-problem}

In the following section the mathematical defintion of the trajectory optimisation problem as proposed by \citeauthor{betts-survey-1998} is discussed \autocite[pp. 1-2]{betts-survey-1998}.

\subsection{Solution Methods for the Trajectory Optimisation Problem}
\label{sec:solution-methods-trajectory-optimisation-problem}

Due to the complexity of the problem when realistical boundary conditions

optimal control

\section{Software Systems for Spacecraft Trajectory Optimisation}
\label{sec:software-systems-for-spacecraft-trajectory-optimisation}

\autocite{conway-elements-2014}

\section{Space Mission Analysis and Design}
\label{sec:space-mission-analysis-and-design}
\citeauthor{wertz-space-1999} divide the \ac{smad} process into four phases with several steps \autocite[p. 2]{wertz-space-1999}:
\begin{itemize}
    \item Define Objectives
        \begin{enumerate}
            \item Define broad objectives and constraints
            \item Estimate quantitative mission needs and requirements
        \end{enumerate}
    \item Characterize the Mission
        \begin{enumerate}[start=3]
            \item Define alternative mission concepts
            \item Define alternative mission architectures
            \item Identify system drivers for each
            \item Characterize mission concepts and architectures
        \end{enumerate}
    \item Evaluate the Mission
        \begin{enumerate}[start=7]
            \item Identify critical requirements
            \item Evaluate mission utility
            \item Define mission concept (baseline)
        \end{enumerate}
    \item Define Requirements
        \begin{enumerate}[start=10]
            \item Define system requirements
            \item Allocate requirements to system elements
        \end{enumerate}
\end{itemize}

\section{Locating a Spacecraft in Space-Time}
\section{The Solar System Environment}
\label{sec:solar-system-environment}
\section{Trajectory Propagation}
\label{sec:trajectory-propagation}
\section{Trajectory Optimisation}
\label{sec:trajectoty-optimisation}

\section{Digital Models}
\label{sec:digital-models}
\autocite{hinsen-computational-2014}
\autocite{hinsen-scientific-2016}
\citeauthor{sussman-role-2002} show that computation can aid the comprehension of the subject matter at hand \autocite{sussman-role-2002}.

\section{Model-Based Engineering}
\label{sec:model-based-engineering}

\subsection{Model-Based Systems Engineering}
\label{sec:model-based-systems-engineering}

