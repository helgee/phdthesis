\chapter{Concept for \topic}
\label{sec:concept}
\todo{Target: 50 pages}

The concept for \topic is developed in the following chapter based on the state of the art as presented in chapter \ref{sec:state-of-the-art} and the requirements as defined in chapter \ref{sec:requirements}.

\section{Overview of the Concept}
\label{sec:overview-concept}

\section{UMLxMD: UML Extension for Multiple Dispatch}
\label{sec:umlxmd}
As discussed in chapter \ref{sec:multiple-dispatch}, using multiple dispatch as the central programming paradigm for an astrodynamics software system offers multiple benefits and is central to this concept.
While it is possible to describe the structure of such a system with a standard UML class diagram, ambiguities might be introduced.
Hence, this thesis proposes a small extension to UML which is based on stereotyping.

Many programming languages that feature multiple dispatch such as Julia (see chapter \ref{sec:julia}) are object-oriented in the sense that every entity is treated as an object.
They do not use classes, though.\footnote{Java on the other hand which heavily features classes and is thus considered an object-oriented language distinguishes between objects and primitives.}
Like other programming languages that have been developed during the last two decades such as Go\todo{citation needed} or Rust\todo{citation needed} they do not bundle data structures and behaviour together into classes but keep them separated into \emph{composite types} and \emph{generic functions}.
\subsection{Composite Types}
\label{sec:composite-types}
A composite data type is a data type that is constructed from other data types.\ct
This thesis follows the convention of the C-family of programming languages in calling composite types \enquote{structures} which is commonly abbreviated as \enquote{struct}.\ct
The equivalent in class-based object-oriented languages would be a class which only has attributes and no methods.
This is called a \enquote{data class} in some languages\ct since it only specifies a data structure and no behaviour.
Figure \ref{fig:struct-uml} shows an \ac{umlxmd} diagram for a \emph{struct}.
\begin{figure}[ht]
    \centering
    \begin{tikzpicture}
        \umlclass[type=struct]{Example}{
            value1 : TypeA \\
            value2 \\
        }{
            (TypeA, Any) \\
            (TypeA) \\
        }
    \end{tikzpicture}
    \caption{\ac{umlxmd} diagram for a composite type.}
    \label{fig:struct-uml}
\end{figure}


The upper section contains the name of the composite type which is a noun and written in camel case\footnote{All words are capitalised without spaces inbetween.} by convention.
To distinguish the \emph{struct} model from a standard \ac{uml} class the \code{<<struct>>} stereotype is added to the type name.

The middle section contains the list of attributes or fields that constitute the composite type together with their associated data types, e.g.~\code{value1} is of type \code{Type}.
If the data type is omited for a field, it is assumed that the value has a dynamic type a runtime, e.g.~in Julia the field would be of type \code{Any}.

The lower section describes additional constructor methods defined by the tuples of the types of their arguments.
In this specific example, the type \code{Example} has the standard constructor with the signature \code{(TypeA, Any)} (i.e.~all fields of the composite type) and an additional constructor with signature \code{(TypeA)} which only takes a single argument and where the second field is computed in the constructor method.
If no additional constructors are specified, the constructor section can be left empty and the existence of the standard constructor is assumed.
Since it is possible to override the standard constructor in some languages, e.g.~Julia, it needs to be explicitly modelled if additional constructor methods are present.

Some multiple dispatch programming languages also support parametric composite types.
Within the composite type definition one or more type parameters are introduced which allow to have one generic definition for families of different types.
Figure \ref{fig:parametric-uml} shows an \ac{umlxmd} diagram for a parametric composite type.
The text in the box with the dashed border in the upper right corner signifies that \code{T} is a type parameter which is the set of all subtypes\footnote{The symbols \code{<:} form the subtype operator in Julia.} of the type \code{Number}.
\begin{figure}[ht]
    \centering
    \begin{tikzpicture}
        \umlclass[type=struct, template={T}]{Struct}{
            value : T \\
        }{}
    \end{tikzpicture}
    \caption{UML stereotype for parametric types.}
    \label{fig:parametric-uml}
\end{figure}


For the example in figure \ref{fig:parametric-uml} this means that \code{value} can be any kind of number while \code{Parametric} is still statically typed.
For example, if \code{value} would be \code{1.0} the type of \code{value} would be \code{Float64} in Julia and the type of the wrapping \code{Parametric} struct would be \code{Parametric{Float64}}.
If \code{value} would be \code{1} on the other hand, the type of \code{value} would be \code{Int64}\footnote{On a 64-bit machine.} and struct type would be \code{Parametric{Int64}}.

\subsection{Generic Functions}
\label{sec:generic-functions}
\begin{figure}[ht]
    \centering
    \begin{tikzpicture}
        \umlclass[type=function]{examplify}{}{
            (TypeA, TypeB) \\
            (TypeA, TypeB, TypeC) \\
            (T) where T <: Number \\
            ... \\
        }
    \end{tikzpicture}
    \caption{UML stereotype for generic functions.}
    \label{fig:function-uml}
\end{figure}


While the middle section could be omitted in theory, it is more practical to leave it empty to ensure compatibility with standard \ac{uml} modelling software tools where this may not be possible.
\subsection{UMLxMD Example}
\label{sec:umlxmd-example}
To illustrate the use of \ac{umlxmd}, a simple model for numbers with associated \ac{si} units is shown in figure \ref{fig:si-example}.

\begin{figure}[ht]
    \centering
    \begin{tikzpicture}
        \begin{umlpackage}{SIUnits}
        \umlclass[type=struct, template={T <: Number}]{UnitNumber}{
            value : T \\
            unit : Symbol \\
        }{}
        \umlclass[below=2 cm of UnitNumber.center, type=function]{+}{}{
            (UnitNumber, T) where T <: Number \\
            (T, UnitNumber) where T <: Number \\
            ... \\
        }
        \end{umlpackage}
    \end{tikzpicture}
    \caption{\ac{umlxmd} example for an SI unit number package.}
    \label{fig:si-example}
\end{figure}


