\chapter{Concept for \topic}
\label{sec:concept}
\todo{Target: 50 pages}

The concept for \topic is developed in the following chapter based on the state of the art as presented in chapter \ref{sec:state-of-the-art} and the requirements as defined in chapter \ref{sec:requirements}.

\section{Overview of the Concept}
\label{sec:overview-concept}

\section{UMLxMD: UML Extension for Multiple Dispatch}
\label{sec:umlxmd}
As discussed in chapter \ref{sec:multiple-dispatch}, using multiple dispatch as the central programming paradigm for an astrodynamics software system offers multiple benefits and is central to this concept.
While it is possible to describe the structure of such a system with a standard UML class diagram, ambiguities might be introduced.
Hence, this thesis proposes a small extension to UML which is based on stereotyping.

Many programming languages that feature multiple dispatch such as Julia (see chapter \ref{sec:julia}) are object-oriented in the sense that every entity is treated as an object.
They do not use classes, though.\footnote{Java on the other hand which heavily features classes and is thus considered an object-oriented language distinguishes between objects and primitives.}
Like other programming languages that have been developed during the last two decades such as Go\todo{citation needed} or Rust\todo{citation needed} they do not bundle data structures and behaviour together into classes but keep them separated into \emph{composite types} and \emph{generic functions}.
\subsection{Composite Types}
\label{sec:composite-types}
\begin{figure}[ht]
    \centering
    \begin{tikzpicture}
        \umlclass[type=struct]{Example}{
            value1 : TypeA \\
            value2 \\
        }{
            (TypeA, Any) \\
            (TypeA) \\
        }
    \end{tikzpicture}
    \caption{\ac{umlxmd} diagram for a composite type.}
    \label{fig:struct-uml}
\end{figure}


\begin{figure}[ht]
    \centering
    \begin{tikzpicture}
        \umlclass[type=struct, template={T}]{Struct}{
            value : T \\
        }{}
    \end{tikzpicture}
    \caption{UML stereotype for parametric types.}
    \label{fig:parametric-uml}
\end{figure}


\subsection{Generic Functions}
\label{sec:generic-functions}
\begin{figure}[ht]
    \centering
    \begin{tikzpicture}
        \umlclass[type=function]{examplify}{}{
            (TypeA, TypeB) \\
            (TypeA, TypeB, TypeC) \\
            (T) where T <: Number \\
            ... \\
        }
    \end{tikzpicture}
    \caption{UML stereotype for generic functions.}
    \label{fig:function-uml}
\end{figure}


\subsection{UMLxMD Example}
\label{sec:umlxmd-example}
To illustrate the use of \ac{umlxmd}, a simple model for numbers with associated \ac{si} units is shown in figure \ref{fig:si-example}.

\begin{figure}[ht]
    \centering
    \begin{tikzpicture}
        \begin{umlpackage}{SIUnits}
        \umlclass[type=struct, template={T <: Number}]{UnitNumber}{
            value : T \\
            unit : Symbol \\
        }{}
        \umlclass[below=2 cm of UnitNumber.center, type=function]{+}{}{
            (UnitNumber, T) where T <: Number \\
            (T, UnitNumber) where T <: Number \\
            ... \\
        }
        \end{umlpackage}
    \end{tikzpicture}
    \caption{\ac{umlxmd} example for an SI unit number package.}
    \label{fig:si-example}
\end{figure}


