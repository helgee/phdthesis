\chapter{Requirements Profile}
\label{sec:requirements-profile}
The requirements profile defined in the following chapter serves to guide the development of the concept for \topic and for the evaluation of the concept's sustainability.

\section{Target definition}
\label{sec:target-definition}

\section{Use Cases}
\label{sec:use-cases}

By identifying the use cases for \ac{mbsmd} the scope of the concept which is developed in this thesis is defined.
The functional and non-functional requirements are then derived from these.
An overview of the considered use cases is given in the diagram in figure \ref{fig:use-cases} which uses the \ac{uml} notation.

\begin{figure}[ht]
    \centering
    \begin{tikzpicture}
        \begin{umlsystem}{Model-Based Space Mission Design}
            \umlusecase[name=design]{Model Mission Design}
            \umlusecase[name=develop, y=-2]{Develop Software Tools}
            \umlusecase[name=validate, y=-4]{Validate Algorithms}
            \umlusecase[name=derive, y=-5.5, width=3cm]{Derive Mission Performance}
            \umlusecase[name=compare, y=-8, width=3cm]{Compare Mission Performance}
        \end{umlsystem}
        \umlactor[x=-5, y=-2]{Mission Analyst}
        \umlactor[x=-5, y=-6]{Systems Engineer}
        \umlactor[x=-5, y=-8]{Architecture Analyst}
        \umlactor[x=-5, y=-4]{Software Developer}
        \umlinclude{compare}{derive}
        \umlinclude{design}{develop}
        \umlextend{validate}{develop}
        \umlassoc{Mission Analyst}{design}
        \umlassoc{Mission Analyst}{develop}
        \umlassoc{Mission Analyst}{validate}
        \umlassoc{Systems Engineer}{derive}
        \umlassoc{Architecture Analyst}{compare}
        \umlassoc{Software Developer}{develop}
        \umlassoc{Software Developer}{validate}
    \end{tikzpicture}
    \caption{Considered use cases in UML notation}
    \label{fig:use-cases}
\end{figure}


\subsection{System Actors}
\label{sec:involved-actors}
The involved system actors as shown in figure \ref{fig:use-cases} are the mission analyst, the software developer, the systems engineer, and the architecture analyst.
\begin{itemize}
    \item \emph{Mission Analyst:}
        The mission analyst defines the trajectory and maneuver strategy of the space mission such that it fulfills its objectives and operational requirements in an optimal way.
        Within \ac{mbsmd} the mission analyst develops the mission model.
        \Heshe uses the model as the basis for mission-specific numerical analysis tools which refine the model and determine derived quantities.
    \item \emph{Software Developer:}
        Within \ac{mbsmd} the development of mission-independent software tools can be outsourced to an external software developer.
        The universal mission model enables \himher to develop inter-operable software solutions.
    \item \emph{Systems Engineer:}
        The systems engineer integrates the mission model into the model of the complete space system.
        \Heshe uses the mission model to derive performance characteristics and ensures that the mission design fulfills the requirements of the overall system.
    \item \emph{Architecture Analyst:}
        The architecture analyst uses collections of mission models to compare the performance of space missions the share an architectural framework or between mission families with different architectures.
\end{itemize}

\subsection{System Boundaries}
\label{sec:system-boundaries}

\subsection{System Behaviour}
\label{sec:system-behaviour}

\section{Requirements}
\label{sec:requirements}
